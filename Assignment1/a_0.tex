% \iffalse
\let\negmedspace\undefined
\let\negthickspace\undefined
\documentclass[journal,12pt,twocolumn]{IEEEtran}
\usepackage{cite}
\usepackage{amsmath,amssymb,amsfonts,amsthm}
\usepackage{algorithmic}
\usepackage{graphicx}
\usepackage{textcomp}
\usepackage{xcolor}
\usepackage{txfonts}
\usepackage{listings}
\usepackage{enumitem}
\usepackage{mathtools}
\usepackage{gensymb}
\usepackage{comment}
\usepackage[breaklinks=true]{hyperref}
\usepackage{tkz-euclide} 
\usepackage{listings}
\usepackage{gvv}                                        
\def\inputGnumericTable{}                                 
\usepackage[latin1]{inputenc}                                
\usepackage{color}                                            
\usepackage{array}                                            
\usepackage{longtable}                                       
\usepackage{calc}                                             
\usepackage{multirow}                                         
\usepackage{hhline}                                           
\usepackage{ifthen}                                           
\usepackage{lscape}

\newtheorem{theorem}{Theorem}[section]
\newtheorem{problem}{Problem}
\newtheorem{proposition}{Proposition}[section]
\newtheorem{lemma}{Lemma}[section]
\newtheorem{corollary}[theorem]{Corollary}
\newtheorem{example}{Example}[section]
\newtheorem{definition}[problem]{Definition}
\newcommand{\BEQA}{\begin{eqnarray}}
\newcommand{\EEQA}{\end{eqnarray}}
\newcommand{\define}{\stackrel{\triangle}{=}}
\theoremstyle{remark}
\newtheorem{rem}{Remark}

\begin{document}

\bibliographystyle{IEEEtran}
\vspace{3cm}

\title{NCERT 11.9.2.3}
\author{EE23BTECH11043 - BHUVANESH SUNIL NEHETE$^{*}$% <-this % stops a space
}
\maketitle
\newpage
\bigskip

\renewcommand{\thefigure}{\theenumi}
\renewcommand{\thetable}{\theenumi}

\bibliographystyle{IEEEtran}

\section*{Question}

In an A.P. the first term is 2 and the sum of the first five terms is one-fourth of the next five terms. Show that 20\textsuperscript{th} term is $-112$.

\section*{Solution}

\begin{table}[h]
    \centering
    \begin{tabular}{|c|c|c|}
        \hline
        \textbf{Parameter} & \textbf{Value/Formula} & \textbf{description}\\
        \hline
        \(x(0)\) & 2 & First term\\
        \hline
        x(19) & -112 & 20\textsuperscript{th} term\\
        \hline
        x(n) & (x(0) + nd)u(n) & $(n+1)\textsuperscript{th}$ term of AP\\
        \hline
        d & -6 & common difference\\
        \hline
    \end{tabular}
    \caption{Input data}
  \label{input data}
\end{table}


   \begin{align}
        x_0 + x_1 + x_2 + x_3 + x_4 = \frac{1}{4} [x_5 + x_6 + x_7 + x_8 + x_9]
    \end{align}

\[[x(0) + x(0) + d + x(0) + 2d + x(0) + 3d + x(0) + 4d)] =\]
\[\frac{1}{4} [x(0) + 5d + x(0) + 6d + x(0) + 7d + x(0) + 8d + x(0) + 9d]\]

Simplifying:
    \begin{align}
        5x(0) + 10d &= \frac{1}{4}(5x(0) + 35d)\\
        \implies x(0) &= \frac{-d}{3}\\
        \implies d &= -6 \label{eq1}
    \end{align}
    \begin{align}
        x(19)&=x(0)+19d\\
        &=2+19(-6) = -112
    \end{align}
From \eqref{eq1} and \tabref{input data}:
    \begin{align}
        \implies x(n)&=(2-6n)u(n)
    \end{align}
    
The Z-transform of x(n) :
    \begin{align}
        X(z) = \sum_{n=1}^{\infty} (2 - 6n)z^{-n}\\
        X(z) = \sum_{n=1}^{\infty} 2z^{-n} - \sum_{n=1}^{\infty} 6nz^{-n}\\
        X(z) = 2U(z) + 6(-z)\frac{d}{dz} U(z)\\
        X(z)=\frac{2}{1-z^{-1}}+\frac{6z^{-1}}{(1-z^{-1})^{2}}
    \end{align}

\[ ROC : |z| > 1\]

    \begin{figure}[h]
        \centering
        \includegraphics[width=1\linewidth]{figs/Figure_1}
        \caption{graph of x(n) = 2 - 6n}
    \end{figure}

    
\end{document}
