% \iffalse
\let\negmedspace\undefined
\let\negthickspace\undefined
\documentclass[journal,12pt,twocolumn]{IEEEtran}
\usepackage{cite}
\usepackage{amsmath,amssymb,amsfonts,amsthm}
\usepackage{algorithmic}
\usepackage{graphicx}
\usepackage{textcomp}
\usepackage{xcolor}
\usepackage{txfonts}
\usepackage{listings}
\usepackage{enumitem}
\usepackage{mathtools}
\usepackage{gensymb}
\usepackage{comment}
\usepackage[breaklinks=true]{hyperref}
\usepackage{tkz-euclide} 
\usepackage{listings}
\usepackage{gvv}                                        
\def\inputGnumericTable{}                                 
\usepackage[latin1]{inputenc}                                
\usepackage{color}                                            
\usepackage{array}                                            
\usepackage{longtable}                                       
\usepackage{calc}                                             
\usepackage{multirow}                                         
\usepackage{hhline}                                           
\usepackage{ifthen}                                           
\usepackage{lscape}

\newtheorem{theorem}{Theorem}[section]
\newtheorem{problem}{Problem}
\newtheorem{proposition}{Proposition}[section]
\newtheorem{lemma}{Lemma}[section]
\newtheorem{corollary}[theorem]{Corollary}
\newtheorem{example}{Example}[section]
\newtheorem{definition}[problem]{Definition}
\newcommand{\BEQA}{\begin{eqnarray}}
\newcommand{\EEQA}{\end{eqnarray}}
\newcommand{\define}{\stackrel{\triangle}{=}}
\theoremstyle{remark}
\newtheorem{rem}{Remark}

\begin{document}

\bibliographystyle{IEEEtran}
\vspace{3cm}

\title{NCERT 11.9.2.3}
\author{EE23BTECH11043 - BHUVANESH SUNIL NEHETE$^{*}$% <-this % stops a space
}
\maketitle
\newpage
\bigskip

\renewcommand{\thefigure}{\theenumi}
\renewcommand{\thetable}{\theenumi}

\bibliographystyle{IEEEtran}

\textbf{Question:}

In an A.P. the first term is 2 and the sum of the first five terms is one-fourth of the next five terms. Show that 20\textsuperscript{th} term is $-112$.

\solution

\begin{table}[h]
    \centering
    \begin{tabular}{|c|c|c|}
        \hline
        \textbf{Parameter} & \textbf{Value/Formula} & \textbf{description}\\
        \hline
        \(x(0)\) & 2 & First term\\
        \hline
        x(19) & -112 & 20\textsuperscript{th} term\\
        \hline
        x(n) & (x(0) + nd)u(n) & $(n+1)\textsuperscript{th}$ term of AP\\
        \hline
        d & -6 & common difference\\
        \hline
    \end{tabular}
    \caption{Input data}
  \label{input data}
\end{table}


General term can be written as
\begin{align}
    x\brak{n} = \brak{x\brak{0} + nd}u\brak{n}
\end{align}
The corresponding Z-transform can be written as 
\begin{align}
    X\brak{z} = \frac{x\brak{0}}{1 - z^{-1}} + \frac{dz^{-1}}{\brak{1 - z^{-1}}^{2}} \label{eq2}
\end{align}
S\brak{n} is the sum of terms from 0 to n,
    \begin{align}
        S\brak{n} &= x\brak{n} * u\brak{n}\\
        S\brak{n-p} &= x\brak{n} * u\brak{n-p}
    \end{align}
On Z-transforming,
    \begin{align}
        S\brak{z} &= X\brak{z}U\brak{z}\\
        S\brak{z} &= \brak{\frac{x\brak{0}}{1 - z^{-1}} + \frac{dz^{-1}}{\brak{1 - z^{-1}}^{2}}}\frac{1}{1-z^{-1}}\\
        S\brak{z} &= \brak{\frac{x\brak{0}}{\brak{1 - z^{-1}}^2} + \frac{dz^{-1}}{\brak{1 - z^{-1}}^{3}}}
    \end{align}
On inverse Z-transforming, 
    \begin{align}
        S\brak{n} &= \oint_{C} S\brak{z}z^{n-1}dz\\
        \implies S\brak{n} &= \oint_{C} \brak{\frac{x\brak{0}}{\brak{1 - z^{-1}}^2} + \frac{dz^{-1}}{\brak{1 - z^{-1}}^{3}}} z^{n-1} dz\\
        \implies S\brak{n} &= \oint_{C} \frac{x\brak{0}z^{2}\brak{z-1}+dz^{2}}{\brak{z-1}^{3}}z^{n-1}dz \label{eq4}
    \end{align}
Using the property, 
    \begin{align}
        f\brak{z_0} &= \oint_{C} \frac{f\brak{z}}{z-z_0}dz\\
        f'\brak{z_0} &= \oint_{C} \frac{f\brak{z}}{\brak{z-z_0}^{2}}dz\\
        f''\brak{z_0} &= 2\oint_{C} \frac{f\brak{z}}{\brak{z-z_0}^{3}}dz \label{eq5}
    \end{align}
On comparison of \eqref{eq4} and \eqref{eq5}
    \begin{align}
        f''\brak{z_0} &= 2S\brak{n}\\
        f\brak{z} &= \brak{x\brak{0}z^2\brak{z-1}+dz^2}z^{n-1}\\
        z_0 &= 1\\
        \implies S\brak{n} &= x\brak{0}\brak{n+1} + \frac{n\brak{n+1}}{2}d
    \end{align}
Given, 
   \begin{align}
       \sum_{n=0}^{4}x\brak{n} = \frac{1}{4}\sum_{n=5}^{9}x\brak{n}
   \end{align}

Simplifying:
    \begin{align}
        S\brak{4} &= \frac{1}{4}\brak{S\brak{9}-S\brak{4}}\\
        5x\brak{0} + 10d &= \frac{1}{4}\brak{5x\brak{0} + 35d}\\
        x\brak{0} &= \frac{-d}{3}\\
        \implies d &= -6 \label{eq1}
    \end{align}
    From \eqref{eq1} and \tabref{input data}
    \begin{align}
        x\brak{19}&=x\brak{0}+19d\\ 
        &= -112
    \end{align}
From \eqref{eq1} and \tabref{input data}:
    \begin{align}
        \implies x\brak{n}&=\brak{2-6n}u\brak{n} \label{eq3}
    \end{align}
    
From \eqref{eq2} and \eqref{eq3} :
    \begin{align}
        X\brak{z}=\frac{2}{1-z^{-1}} - \frac{6z^{-1}}{\brak{1-z^{-1}}^{2}} \quad |z|>1
    \end{align}

    \begin{figure}[h]
    \renewcommand\thefigure{1}
        \centering
        \includegraphics[width=1\linewidth]{figs/Figure_1.png}
        \caption{graph of x\brak{n} = 2 - 6n}
    \end{figure}

\end{document}
