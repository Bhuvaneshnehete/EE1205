% \iffalse
\let\negmedspace\undefined
\let\negthickspace\undefined
\documentclass[journal,12pt,twocolumn]{IEEEtran}
\usepackage{cite}
\usepackage{amsmath,amssymb,amsfonts,amsthm}
\usepackage{algorithmic}
\usepackage{graphicx}
\usepackage{textcomp}
\usepackage{xcolor}
\usepackage{txfonts}
\usepackage{listings}
\usepackage{enumitem}
\usepackage{mathtools}
\usepackage{gensymb}
\usepackage{comment}
\usepackage[breaklinks=true]{hyperref}
\usepackage{tkz-euclide} 
\usepackage{listings}
\usepackage{gvv}                                        
\def\inputGnumericTable{}                                 
\usepackage[latin1]{inputenc}                                
\usepackage{color}                                            
\usepackage{array}                                            
\usepackage{longtable}                                       
\usepackage{calc}                                             
\usepackage{multirow}                                         
\usepackage{hhline}                                           
\usepackage{ifthen}                                           
\usepackage{lscape}

\newtheorem{theorem}{Theorem}[section]
\newtheorem{problem}{Problem}
\newtheorem{proposition}{Proposition}[section]
\newtheorem{lemma}{Lemma}[section]
\newtheorem{corollary}[theorem]{Corollary}
\newtheorem{example}{Example}[section]
\newtheorem{definition}[problem]{Definition}
\newcommand{\BEQA}{\begin{eqnarray}}
\newcommand{\EEQA}{\end{eqnarray}}
\newcommand{\define}{\stackrel{\triangle}{=}}
\theoremstyle{remark}
\newtheorem{rem}{Remark}

\begin{document}

\bibliographystyle{IEEEtran}
\vspace{3cm}

\title{GATE EE23 27}
\author{EE23BTECH11043 - BHUVANESH SUNIL NEHETE$^{*}$% <-this % stops a space
}
\maketitle
\newpage
\bigskip

\renewcommand{\thefigure}{\theenumi}
\renewcommand{\thetable}{\theenumi}

\bibliographystyle{IEEEtran}

\textbf{Question:}
Which of the following statement(s) is/are true?
\begin{enumerate}[label=\alph*)]
    \item If an LTI system is causal, it is stable
    \item A discrete time LTI system is causal if and only if its response to a step input $u[n]$ is 0 for $n < 0$
    \item If a discrete time LTI system has an impulse response $h[n]$ of finite duration the system is stable
    \item If the impulse response $0<|h[n]|<1$ for all $n$, then the LTI system is stable.
\end{enumerate}

\solution
\begin{enumerate}
    \item A system is said to be causal if its output at any given time depends only on past and present inputs, not future inputs. Mathematically, a system is causal if its impulse response is zero for negative time.\\
    On the other hand, a system is considered stable if its output remains bounded for bounded inputs.\\
    So, it is not always necessary for an LTI system if it is causal, then is stable.
    \item In discrete-time signal processing, a system is considered causal if the output at any time depends only on the current and past inputs, not future inputs. This means that the response of the system to a step input (a signal that is zero for negative time indices and unity for non-negative time indices) should also be zero for negative time indices if the system is causal.\\
    Conversely, if the response to a step input is non-zero for negative time indices, it implies that the system is anticipating future inputs, which violates the causality principle.\\
    So, a discrete-time LTI system is causal if and only if its response to a step input $u[n]$ is 0 for $n<0$.
    \item An LTI system is said to be stable if it follows BIBO rule (output should be bounded for bounded input) but in option \brak{C} output of h[n] is not defined, so it not necessarily to be stable.
    \item An LTI system is said to be stable if it follows BIBO rule (output should be bounded for bounded input) but in option \brak{D} range of n is not defined, so it is not necessarily to be stable.
\end{enumerate}

u\end{document}
